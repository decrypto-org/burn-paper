\section{Analysis}

We now move on to the analysis of our scheme. As the scheme is deterministic,
its correctness is straightforward to show.

\begin{theorem}[Correctness]
  The proof-of-burn protocol $\Pi$ of Section~\ref{sec:construction} is \emph{correct}.
\end{theorem}
\begin{proof}
  Based on Algorithm~\ref{alg.construction-ro}, $\BurnVerify(1^\kappa, t, \GenBurnAddr(1^\kappa, t)) = \textsf{true}$ if and only if $\GenBurnAddr(1^\kappa, t) = \GenBurnAddr(1^\kappa, t)$, which always holds as $\GenBurnAddr$ is deterministic.
\end{proof}

We now state a simple lemma pertaining to the distribution of Random Oracle
outputs.

\begin{restatable}[Perturbation]{lem}{restateLemPerturbation}
  \label{lem.perturbation}
  Let $p(\kappa)$ be a polynomial and
  $F: \{0,1\}^\kappa \longrightarrow \{0,1\}^\kappa$ be a permutation.
  Consider the process which samples $p(\kappa)$ strings $s_1, s_2, \dots, s_{p(\kappa)}$ uniformly at random from the set $\{0, 1\}^\kappa$. The probability that there exists $i \neq j$ such that $s_i = F(s_j)$ is negligible in $\kappa$.
\end{restatable}

We will now apply the above lemma to show that our scheme is unspenable.

\begin{theorem}[Unspendability]
  If $H$ is a \emph{Random Oracle}, then the protocol $\Pi$ of Section~\ref{sec:construction} is \emph{unspendable}.
\end{theorem}
\begin{proof}
  Let $\mathcal{A}$ be an arbitrary probabilistic polynomial time $\spendattack$ adversary.
  The adversary $\mathcal{A}$ makes at most a polynomial number of queries $p(\kappa)$ to the Random Oracle.
  Consider the event \textsc{Match} of Lemma~\ref{lem.perturbation}.

  If the adversary is successful then it has presented $t, pk, pkh$ such that $H(pk) = pkh$ and $H(t) \xor 1 = pkh$.
  We observe that $\spendattack_{\mathcal{A}, \Pi}(\kappa) = \true \Rightarrow \textsc{Match}$.
  Therefore $\Pr[\spendattack_{\mathcal{A}, \Pi}(\kappa)] \leq Pr[\textsc{Match}]$. Applying Lemma~\ref{lem.perturbation} for the permutation $F(x) = x \xor 1$,
  we obtain
  $\Pr[\spendattack_{\mathcal{A}, \Pi}(\kappa)] \leq \negl$.
\end{proof}

We note that the security of the signature scheme is not needed to prove unspendability. Were the signature scheme of the underlying cryptocurrency ever found to be \emph{forgeable}, the coins burned through our scheme would remain unspendable. We additionally remark that the
choice of the permutation $F(x) = x \xor 1$ is arbitrary. Any one-to-one
function beyond the identity function would work equally well.

Next, our binding theorem only requires that the hash function used is collision
resistant and is in the standard model.

\import{./}{algorithms/alg.collision-adversary.tex}

\begin{theorem}[Binding]
  If $H$ is a \emph{collision resistant} hash function then the protocol of Section~\ref{sec:construction} is \emph{binding}.
\end{theorem}
\begin{proof}
  Let $\mathcal{A}$ be an arbitrary adversary against $\Pi$.
  We will construct the Collision Resistance adversary $\mathcal{A}^*$ against $H$.

  The collision resistance adversary, illustrated in Algorithm~\ref{alg.collision-adversary}, calls $\mathcal{A}$ and obtains two outputs, $t$ and $t'$. If $\mathcal{A}$ is successful then $t \neq t'$ and $H(t) \xor 1 = H(t') \xor 1$. Therefore $H(t) = H(t')$.

  We thus conclude that $\mathcal{A^*}$ is successful in the $\collisionattack$ game if and only if $\mathcal{A}$ is successful in the $\bindattack$ game.

  \[
    \Pr[\bindattack_{\mathcal{A},\Pi}(\kappa) = \true]
    =
    \Pr[\collisionattack_{\mathcal{A}^*,H}(\kappa) = \true]
  \]

  From the collision resistance of $H$ it follows that $\Pr[\collisionattack_{\mathcal{A}^*,H} = 1] < \negl$. Therefore,
  $\Pr[\bindattack_{\mathcal{A},\Pi} = 1] < \negl$, so
  the protocol $\Pi$ is binding.
\end{proof}

We now posit that no adversary can predict the public key of a secure signature scheme, except with negligible probability. We call a distribution \emph{unpredictable} if no
probabilistic polynomial-time adversary can predict its sampling. We give
the formal definition, with some of its statistical properties, in
Appendix~\ref{sec.background}.

% TODO(dionyziz): If this pf is moved to appendix, write out \mathcal{A} as algorithm
\begin{restatable}[Public key unpredictability]{lem}{restateLemPkUnpredictability}
  \label{lem:pk-unpredictability}
  Let $S = (\textsf{Gen}, \textsf{Sig}, \textsf{Ver})$ be a secure signature scheme.
  Then the distribution ensemble
  $X_\kappa = \{(sk, pk) \gets \textsf{Gen}(1^\kappa); pk\}$ is
  unpredictable.
\end{restatable}

The following theorem shows that the output of the random oracle is
indistinguishable from random if the input is unpredictable. For reference, the
definition of computational indistinguishability is included in
Appendix~\ref{sec.comp-ind}.

\begin{restatable}[Random Oracle unpredictability]{lem}{restateLemRoUnpredictability}
  \label{lem:ro-unpredictability}
  Let $\mathcal{T}$ be an unpredictable distribution ensemble and $H$ be a
  Random Oracle.
  The distribution $X = \{t \gets \mathcal{T}; H(t)\}$ is indistinguishable from
  the uniform distribution $\uniformk$.
\end{restatable}

\begin{theorem}[Uncensorability]
  Let $S = (\Gen, \Sig, \Ver)$ be a \emph{secure signature scheme},
  $H$ be a \emph{Random Oracle},
  and $\mathcal{T}$ be an unpredictable tag distribution.
  Then the protocol of Section~\ref{sec:construction} instantiated with
  $H, S, \mathcal{T}$ is \emph{uncensorable}.
\end{theorem}
\begin{proof}
  Let $X$ be the distribution ensemble of public keys generated using $\GenAddr$
  and $Y$ that of keys generated using $\GenBurnAddr$.

  From Lemma~\ref{lem:pk-unpredictability} the distribution of
  public keys generated from $S$ is unpredictable. The
  function $\GenAddr$ samples a public key from $S$ and applies the
  random oracle $H$ to it. Applying
  Lemma~\ref{lem:ro-unpredictability}, we obtain that
  $X \cind \uniform(\{0, 1\}^\kappa)$.

  The function $H'(x) = H(x) \xor 1$ is a random oracle (despite not
  being independent from the random oracle $H$).
  Since $\mathcal{T}$ is unpredictable, and
  applying Lemma~\ref{lem:ro-unpredictability} with random oracle $H'$, we
  obtain that $Y \cind \uniform(\{0, 1\}^\kappa)$.

  By transitivity, $X$ and $Y$ are computationally indistinguishable.
\end{proof}

From the above, we conclude that the tags used during the burn process must be
unpredictable. If the tag is chosen to contain a randomly generated public key
from a secure signature scheme, or its hash,
Lemmas~\ref{lem:pk-unpredictability}~and~\ref{lem:ro-unpredictability} show that
sufficient entropy exists to ensure uncensorability. Our cross-chain application
makes use of this fact.
