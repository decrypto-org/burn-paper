\section{Analysis}

\begin{theorem}[Correctness]
  The Proof-of-Burn protocol $\Pi$ of Section~\ref{section:construction} is \emph{correct}.
\end{theorem}
\begin{proof}
  We need to prove that $\forall \kappa, t. \BurnVerify(1^\kappa, t, \GenBurnAddress(1^\kappa, t)) = \textsf{true}$.

  Based on Algorithm~\ref{alg.construction}, $\BurnVerify(1^\kappa, t, \GenBurnAddress(1^\kappa, t)) = \textsf{true}$ if and only if $\GenBurnAddress(1^\kappa, t) = \GenBurnAddress(1^\kappa, t)$, which always holds as $\GenBurnAddress$ is deterministic.
\end{proof}

\begin{lemma}[Perturbation]
  \label{lem.perturbation}
  Let $p(\kappa)$ be a polynomial. Consider the process which samples $p(\kappa)$ strings $s_1, s_2, \dots, s_{p(\kappa)}$ uniformly at random from the set $\{0, 1\}^\kappa$. The probability that there exists $i$, $j$ such that $s_i = s_j \xor 1$ is negligible in $\kappa$.
\end{lemma}
\begin{proof}
  Let \textsc{Match} denote the event that there exist $1 \leq i, j \leq p(\kappa)$ such that $s_i = s_j \xor 1$.
  Let $\textsc{Match}_{i, j}$ denote the event that $s_i = s_j \xor 1$. Apply a union bound to obtain

  \begin{align*}
    \Pr[\bigcup_{i, j}\textsc{Match}_{i, j}] &\leq \sum_{i, j} \Pr[\textsc{Match}_{i, j}] \\
    \Pr[\textsc{Match}] &\leq \sum_{i, j} \Pr[\textsc{Match}_{i, j}]
  \end{align*}

  We observe that if $i \neq j$ then $\Pr[\textsc{Match}_{i, j}] = 2^{-p(\kappa)}$, otherwise $\Pr[\textsc{Match}_{i, j}] = 0$. Therefore, $\Pr[\textsc{Match}] \leq \sum_{i, j} 2^{-p(\kappa)} = p^2(\kappa) 2^{-p(\kappa)}$.
\end{proof}

\begin{theorem}[Unspendability]
  If $H$ is a \emph{Random Oracle}, then the protocol $\Pi$ of Section~\ref{section:construction} is \emph{unspendable}.
\end{theorem}
\begin{proof}
  Let $\mathcal{A}$ be an arbitrary probabilistic polynomial time \textsc{spend-attack} adversary.
  The adversary $\mathcal{A}$ makes at most a polynomial number of queries $p(\kappa)$ to the Random Oracle.
  Consider the event \textsc{Match} of Lemma~\ref{lem.perturbation}.

  If the adversary is successful then it has presented $t, pk, pkh$ such that $H(pk) = pkh$ and $H(t) \xor 1 = pkh$.

  We observe that $\textsc{spend-attack}_{\mathcal{A}, \Pi}(\kappa) = \true \Rightarrow \textsc{Match}$.

  Therefore $\Pr[\textsc{spend-attack}_{\mathcal{A}, \Pi}(\kappa)] \leq Pr[\textsc{Match}]$. Applying Lemma~\ref{lem.perturbation}, $\Pr[\textsc{spend-attack}_{\mathcal{A}, \Pi}(\kappa)] \leq negl(\kappa)$.
\end{proof}

We note that the security of the signature scheme is not needed to prove unspendability. Were the signature scheme of Bitcoin ever found to be forgeable, the coins burned through our scheme would remain unspendable.

\import{./}{algorithms/alg.collision-resistance-adversary.tex}

\begin{theorem}[Binding]
  If $H$ is a \emph{collision resistant} hash function then the protocol of Section~\ref{section:construction} is \emph{binding}.
\end{theorem}
\begin{proof}
  Let $\mathcal{A}$ be an arbitrary adversary against $\Pi$.
  We will construct the Collision Resistance adversary $\mathcal{A}^*$ against $H$.

  The collision resistance adversary, illustrated in Algorithm~\ref{alg.collision-resistance-adversary}, calls $\mathcal{A}$ and obtains two outputs, $t$ and $t'$. If $\mathcal{A}$ is successful then $H(t) \xor 1 = H(t') \xor 1$ and therefore $H(t) = H(t')$.

  We thus conclude that $\mathcal{A^*}$ is successful in the \textsf{Collision} game if and only if $\mathcal{A}$ is successful in the \textsf{Bind} game.

  \[
    \Pr[\textsf{Bind}_{\mathcal{A},\Pi} = \true]
    =
    \Pr[\textsf{Collision}_{\mathcal{A}^*,H} = \true]
  \]

  From the collision resistance of $H$ it follows that $\Pr[\textsf{Collision}_{\mathcal{A}^*,H} = 1] < negl(\kappa)$. Therefore,
  $\Pr[\textsf{Bind}_{\mathcal{A},\Pi} = 1] < negl(\kappa)$, so
  the protocol $\Pi$ is binding.
\end{proof}

% TODO: This lemma seems useless... We need to argue about predictability of Gen() by the adversary.
\begin{lemma}[Distinct keys]\label{lem:distinct-keys}
  Let $S = (\Gen, \Sig, \Ver)$ be a secure signature scheme and $p$ be any polynomial. Consider the process which calls $(pk_i, sk_i) \gets \Gen(1^\kappa)$ repeatedly and independently $p(\kappa)$ times to obtain $p(\kappa)$ samples. Then for all $i, j \in [p]$ with $i \neq j$, we have that $pk_i \neq pk_j$, except with negligible probability in $\kappa$.
\end{lemma}
\begin{proof}
  Consider the above process and let \textsc{repeat-key} be the event of two samples $i \neq j$ repeating, i.e., $pk_i = pk_j$. Then consider the following existential forgery adversary $\mathcal{A}$ for the signature scheme $S$. The adversary receives a public key $pk$ and attempts to forge a signature $\sigma$. The adversary calls $(sk', pk') \gets \Gen(1^\kappa)$ to generate a new key $(sk', pk')$. If $pk' \neq pk$, the adversary aborts. Otherwise, the adversary uses $sk'$ to create a signature forgery.

  If $pk = pk'$ then by the correctness of the signature scheme the forgery will be successful. Note that \textsc{repeat-key} is the event of any two keys being repeated among $p$ keys. Additionally, together the forgery challenger $\textsf{Sig-forge}$ and the adversary $\mathcal{A}$ independently generate a pair of keys. Then by applying a union bound we obtain that $p^2 \Pr[\textsf{Sig-forge}^{cma}_{\mathcal{A},S}] \geq \Pr[\textsc{repeat-key}]$. From the fact that $p$ is a polynomial and
  $\Pr[\textsf{Sig-forge}^{cma}_{\mathcal{A},S}]$ is negligible, it follows that $\Pr[\textsc{repeat-key}]$ is negligible.
\end{proof}

\begin{theorem}
  Let $H$ be a \emph{Random Oracle}, $S = (\Gen, \Sig, \Ver)$ be a \emph{secure signature scheme} and $p$ be any polynomial with $p(\kappa) \in \Omega(\kappa)$. Consider the tag distribution $\mathcal{T}$ to be the uniform distribution of the set $\{0, 1\}^{p(\kappa)}$.
  Then the protocol of Section~\ref{section:construction} is \emph{uncensorable}.
\end{theorem}
\begin{proof}
  We will argue that both
  $X = \{x: x \gets \textsf{GenAddress}\}$ and
  $Y = \{y: t \gets \mathcal{T}; y \gets \textsf{GenBurnAddress}(1^\kappa, t)\}$ are distributions computationally indistinguishable from the uniform random distribution sampled from $\{0, 1\}^\kappa$. Therefore $X$ and $Y$ will be computationally indistinguishable.

  For $X$, the distinguisher $\mathcal{A}$ receives a truly random input $pkh = H(pk)$ which they can only distinguish if they are able to create a query to the random oracle with the same $pk$. However, we know from Lemma~\ref{lem:distinct-keys} that no two queries to $Gen(1^\kappa)$ will return the same value except with negligible probability. Therefore the adversary will not make the random oracle query $pk$, and so $x$ is indistinguishable from uniformly random.
  % TODO: This seems irrelevant... the adversary could generate pk without calling Gen(1^\kappa)...

  For $Y$, the distinguisher $\mathcal{A}$ receives a truly random input $pkh = H(t)$ where $t$ is sampled from $\mathcal{T}$. From the fact that $p \in \Omega(\kappa)$, the value $t$ is unpredictable to the adversary exept with negligible probability. Therefore the adversary will not make the query $t$ to the random oracle and its input $y$ is indistinguishable from uniformly random.
  % TODO: This argument seems shaky... we need to make it more formal.
\end{proof}
