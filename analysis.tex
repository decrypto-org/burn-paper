\section{Analysis}

We now move on to the analysis of our scheme. As the scheme is deterministic,
its correctness is straightforward to show.

\begin{theorem}[Correctness]
  The proof-of-burn protocol $\Pi$ of Section~\ref{sec:construction} is \emph{correct}.
\end{theorem}
\begin{proof}
  Based on Algorithm~\ref{alg.construction-ro}, $\BurnVerify(1^\kappa, t, \GenBurnAddr(1^\kappa, t)) = \textsf{true}$ if and only if $\GenBurnAddr(1^\kappa, t) = \GenBurnAddr(1^\kappa, t)$, which always holds as $\GenBurnAddr$ is deterministic.
\end{proof}

We now state a simple lemma pertaining to the distribution of Random Oracle
outputs.

\begin{lemma}[Perturbation]
  \label{lem.perturbation}
  Let $p(\kappa)$ be a polynomial and
  $F: \{0,1\}^\kappa \longrightarrow \{0,1\}^\kappa$ be a permutation.
  Consider the process which samples $p(\kappa)$ strings $s_1, s_2, \dots, s_{p(\kappa)}$ uniformly at random from the set $\{0, 1\}^\kappa$. The probability that there exists $i \neq j$ such that $s_i = F(s_j)$ is negligible in $\kappa$.
\end{lemma}
\begin{proof}
  Let \textsc{Match} denote the event that there exist $1 \leq i \neq j \leq p(\kappa)$ such that $s_i = F(s_j)$.
  Let $\textsc{Match}_{i, j}$ denote the event that $s_i = F(s_j)$. Apply a union bound to obtain
  $
    \Pr[\bigcup_{i, j}\textsc{Match}_{i, j}] \leq \sum_{i, j} \Pr[\textsc{Match}_{i, j}]
    \Pr[\textsc{Match}] \leq \sum_{i, j} \Pr[\textsc{Match}_{i, j}]
  $.
  But $\Pr[\textsc{Match}_{i, j}] = 2^{-p(\kappa)}$ and therefore
  $\Pr[\textsc{Match}] \leq \sum_{i \neq j} 2^{-p(\kappa)} \leq p^2(\kappa) 2^{-p(\kappa)}$.
\end{proof}

We will now apply the above lemma to show that our scheme is unspenable.

\begin{theorem}[Unspendability]
  If $H$ is a \emph{Random Oracle}, then the protocol $\Pi$ of Section~\ref{sec:construction} is \emph{unspendable}.
\end{theorem}
\begin{proof}
  Let $\mathcal{A}$ be an arbitrary probabilistic polynomial time $\spendattack$ adversary.
  The adversary $\mathcal{A}$ makes at most a polynomial number of queries $p(\kappa)$ to the Random Oracle.
  Consider the event \textsc{Match} of Lemma~\ref{lem.perturbation}.

  If the adversary is successful then it has presented $t, pk, pkh$ such that $H(pk) = pkh$ and $H(t) \xor 1 = pkh$.
  We observe that $\spendattack_{\mathcal{A}, \Pi}(\kappa) = \true \Rightarrow \textsc{Match}$.
  Therefore $\Pr[\spendattack_{\mathcal{A}, \Pi}(\kappa)] \leq Pr[\textsc{Match}]$. Applying Lemma~\ref{lem.perturbation} for the permutation $F(x) = x \xor 1$,
  we obtain
  $\Pr[\spendattack_{\mathcal{A}, \Pi}(\kappa)] \leq \negl$.
\end{proof}

We note that the security of the signature scheme is not needed to prove unspendability. Were the signature scheme of the underlying cryptocurrency ever found to be \emph{forgeable}, the coins burned through our scheme would remain unspendable. We additionally remark that the
choice of the permutation $F(x) = x \xor 1$ is arbitrary. Any one-to-one
function beyond the identity function would work equally well.

Next, our binding theorem only requires that the hash function used is collision
resistant and is in the standard model.

\import{./}{algorithms/alg.collision-resistance-adversary.tex}

\begin{theorem}[Binding]
  If $H$ is a \emph{collision resistant} hash function then the protocol of Section~\ref{sec:construction} is \emph{binding}.
\end{theorem}
\begin{proof}
  Let $\mathcal{A}$ be an arbitrary adversary against $\Pi$.
  We will construct the Collision Resistance adversary $\mathcal{A}^*$ against $H$.

  The collision resistance adversary, illustrated in Algorithm~\ref{alg.collision-resistance-adversary}, calls $\mathcal{A}$ and obtains two outputs, $t$ and $t'$. If $\mathcal{A}$ is successful then $t \neq t'$ and $H(t) \xor 1 = H(t') \xor 1$. Therefore $H(t) = H(t')$.

  We thus conclude that $\mathcal{A^*}$ is successful in the \textsf{Collision} game if and only if $\mathcal{A}$ is successful in the \textsf{Bind} game.

  \[
    \Pr[\textsf{Bind}_{\mathcal{A},\Pi} = \true]
    =
    \Pr[\textsf{Collision}_{\mathcal{A}^*,H} = \true]
  \]

  From the collision resistance of $H$ it follows that $\Pr[\textsf{Collision}_{\mathcal{A}^*,H} = 1] < \negl$. Therefore,
  $\Pr[\textsf{Bind}_{\mathcal{A},\Pi} = 1] < \negl$, so
  the protocol $\Pi$ is binding.
\end{proof}

Recall that a distribution ensemble is \emph{unpredictable} if no
polynomial-time adversary can guess its output. The cryptographic
predictability game is illustrated in Algorithm~\ref{alg.predict-game}.

\import{.}{./algorithms/alg.predict-game.tex}

\begin{definition}[Unpredictable distribution]
  A distribution ensemble $\{X_{\kappa}\}_{\kappa\in\mathbb{N}}$ is
  \emph{unpredictable} if for all probabilistic polynomial-time adversaries
  $\mathcal{A}$ there exists a negligible function $\negl$ such that
  $\Pr[\textsc{predict}_{\mathcal{A},X}(\kappa) = \true] < \negl$.
\end{definition}

\begin{lemma}[Negligible predictability]\label{lem:negl-pred}
  Consider a distribution ensemble $\{X_{\kappa}\}_{\kappa\in\mathbb{N}}$ and
  a negligible function $\negl$. If for all $\kappa$ for all
  $x \in [X_\kappa]$ we have that
  $\Pr_{x^* \gets X_\kappa}[x^* = x] \leq \negl$, then
  $X$ is unpredictable.
\end{lemma}
\begin{proof}
  Consider a probabilistic polynomial-time adversary $\mathcal{A}$ which
  predicts $X_\kappa$. The adversary is not given any input beyond $1^\kappa$,
  hence the distribution of its output is independent from the choice of the
  challenger. Therefore

  \begin{align*}
  \Pr[\textsc{predict}_{\mathcal{A},X}(\kappa) = \true] &=
  \sum_{x' \in [X]}\Pr_{x \gets X}[\mathcal{A}(\kappa) = x'\land x = x'] =\\
  \sum_{x' \in [X]}\Pr[\mathcal{A}(\kappa) = x']\Pr_{x \gets X}[x = x']
  &\leq \negl\sum_{x' \in [X]}\Pr[\mathcal{A}(\kappa) = x']
  \leq \negl
  \end{align*}
\end{proof}

We now posit that no adversary can predict the public key of a secure signature scheme, except with negligible probability.

\begin{lemma}[Public key unpredictability]\label{lem:pk-unpredictability}
  Let $S = (\textsf{Gen}, \textsf{Sig}, \textsf{Ver})$ be a secure signature scheme.
  Then the distribution ensemble
  $X_\kappa = \{(sk, pk) \gets \textsf{Gen}(1^\kappa); pk\}$ is
  unpredictable.
\end{lemma}
\begin{proof}
  Let $\widehat{pk}$ be an arbitrary $\kappa$-length public key generated
  by $\textsf{Gen}$ with some probability $p$.
  Consider the existential forgery adversary $\mathcal{A}$ which works as
  follows. It receives $pk$ as its input from the challenger, but ignores it
  and generates a new key pair $(pk', sk') \gets \textsf{Gen}(1^\kappa)$.
  Since the
  two invocations of $\textsf{Gen}$ are independent,
  $\Pr[pk = pk'] \geq \Pr[pk = \widehat{pk} \land pk' = \widehat{pk}]
  = \Pr[pk = \widehat{pk}]\Pr[pk = \widehat{pk'}] = p^2$. The adversary checks
  whether $pk = pk'$. If not, it aborts. Otherwise, it uses $sk'$ to sign the
  message $m = \epsilon$ and returns the forgery $\sigma = \textsf{Sig}(sk, m)$.
  From the correctness of the signature scheme, if $pk = pk'$, then
  $\textsf{Ver}(pk, \textsf{Sig}(sk, m)) = \true$ and the adversary is
  successful. Since the signature scheme is secure,
  $\Pr[\textsf{Sig-forge}^{cma}_{\mathcal{A},S}] = \negl$.
  But $\Pr[pk = pk'] \leq \Pr[\textsf{Sig-forge}^{cma}_{\mathcal{A},S}]$ and
  therefore $p = \sqrt{\Pr[pk = pk']} \leq \negl$. Applying
  Lemma~\ref{lem:negl-pred}, we deduce that the distribution $X$ is
  unpredictable.
\end{proof}

\import{.}{./algorithms/alg.predictability-adversary.tex}

The following theorem shows that the output of the random oracle is
indistinguishable from random if the input is unpredictable. For reference, the
definition of computational indistinguishability is included in
Appendix~\ref{sec.background}.

\begin{lemma}[Random Oracle unpredictability]\label{lem:ro-unpredictability}
  Let $\mathcal{T}$ be an unpredictable distribution ensemble and $H$ be a
  Random Oracle.
  The distribution $X = \{t \gets \mathcal{T}; H(t)\}$ is indistinguishable from
  the uniform distribution $\uniformk$.
\end{lemma}
\begin{proof}
  Let $\mathcal{A}$ be an arbitrary polynomial distinguisher between
  $X$ and $\uniform(\{0, 1\}^\kappa)$.
  We construct an adversary $\mathcal{A}^*$
  against $\textsc{predict}_{\mathcal{T}}$.
  Let $r$ denote the (polynomial)
  maximum number of random oracle queries of $\mathcal{A}$.
  The adversary $\mathcal{A}^*$ is illustrated in
  Algorithm~\ref{alg.predictability-adversary} and works as follows.
  Initially, it chooses a random bit $b \stackrel{\$}{\gets} \{0, 1\}$ and
  sets $Z = X$ if $b = 0$, otherwise
  sets $Z = \uniformk$.
  It samples $z \gets Z$.
  If $b = 0$, then $z$ is chosen by applying $\GenAddr$ which involves
  calling the random oracle $H$ with some input $pk$.
  It then chooses one of $\mathcal{A}$'s queries $j \stackrel{\$}{\gets} [r]$
  uniformly at random. Finally, it outputs the input received by the random
  oracle during the $j^\text{th}$ query of $\mathcal{A}$.

  We will consider two cases. Either $\mathcal{A}$ makes a random oracle query
  containing $pk$, or it does not. We will argue that, if $\mathcal{A}$ makes
  a random oracle query containing $pk$ with non-negligible probability, then
  $\mathcal{A}^*$ will be successful with non-negligible probability. However,
  we will argue that, if $\mathcal{A}$ does not make the particular random
  oracle query, it will be unable to distinguish $X$ from $\uniformk$.

  Let $\query$ denote the event that $b = 0$ and $\mathcal{A}$ asks a random
  oracle query with input $pk$.
  Let $x$ denote the random variable sampled by the challenger in the
  predictability game of $\mathcal{A}^*$.
  Let $\extqry$ denote the event that $b = 0$ and $\mathcal{A}$ asks a
  random oracle query with input equal to $x$. Observe that, since the input to
  $\mathcal{A}$ does not depend on $x$, we have that
  $\Pr[\extqry] = \Pr[\query]$. As $j$ is chosen independently of the execution
  of $\mathcal{A}$, conditioned on $\extqry$ the probability that
  $\mathcal{A}^*$ is able to correctly guess which query caused $\extqry$ will
  be $\frac{1}{r}$. Therefore we obtain that
  $\Pr[\textsc{predict}_{\mathcal{A}^*,\mathcal{T}}(\kappa) = \true]
   = \frac{1}{r}\Pr[\extqry]
   = \frac{1}{r}\Pr[\query]$.
  As
  $\Pr[\textsc{predict}_{\mathcal{A}^*,\mathcal{T}}(\kappa) = \true] \leq \negl$
  and $r$
  is polynomial in $\kappa$, we deduce that $\Pr[\query] \leq \negl$.

  Consider the computational
  indistinguishability game depicted in
  Algorithm~\ref{alg.dist-game} in which the distinguisher gives a guess $b^*$
  attempting to identify the origin $b$ of its input.
  If $b = 0$, then the distinguisher $\mathcal{A}$ receives a truly random input
  $pkh = H(pk)$.
  If the distinguisher does not query the random oracle
  with input $pk$, the input of the distinguisher is truly random
  and therefore $\Pr[b^* = 0|b = 0|\lnot \query] = \Pr[b^* = 0|b = 1]$.

  Consider the case where $b = 0$ and apply total probability to obtain
  \begin{align*}
    &\Pr[b^* = 0|b = 0] =\\
    &\Pr[b^* = 0|\query]\Pr[\query] +
      \Pr[b^* = 0|b = 0|\lnot \query]\Pr[\lnot \query]\\
    \leq &\Pr[b^* = 0|\query]\Pr[\query] +
      \Pr[b^* = 0|b = 0|\lnot \query]\\
    \leq &\Pr[\query] + \Pr[b^* = 0|b = 0|\lnot \query]
  \end{align*}

  Then
  $
    \Pr[\textsc{dist-game}_{\mathcal{A},X,\uniformk} = \true]
    =
    \Pr[b = b^*]
  $ is the probability of success of the distinguisher.
  Applying total probability we obtain

  \begin{align*}
    \Pr[b = b^*] &= \Pr[b = b^*|b = 0]\Pr[b = 0] + \Pr[b = b^*|b = 1]\Pr[b = 1]\\
                 &= \frac{1}{2}(\Pr[b^* = 0|b = 0] + \Pr[b^* = 1|b = 1])\\
                 &\leq \frac{1}{2}(\Pr[\query] + \Pr[b^* = 0|b = 0|\lnot \query]
                 + \Pr[b^* = 1|b = 1])\\
                 &= \frac{1}{2}(\Pr[\query] + \Pr[b^* = 0|b = 1]
                 + \Pr[b^* = 1|b = 1])\\
                 &= \frac{1}{2}(\Pr[\query] + \Pr[b^* = 0|b = 1]
                 + (1 - \Pr[b^* = 0|b = 1]))\\
                 &= \frac{1}{2}(1 + \Pr[\query]) \leq \frac{1}{2} + \negl
  \end{align*}
\end{proof}

\begin{theorem}[Uncensorability]
  Let $S = (\Gen, \Sig, \Ver)$ be a \emph{secure signature scheme},
  $H$ be a \emph{Random Oracle},
  and $\mathcal{T}$ be an unpredictable tag distribution.
  Then the protocol of Section~\ref{sec:construction} instantiated with
  $H, S, \mathcal{T}$ is \emph{uncensorable}.
\end{theorem}
\begin{proof}
  From Lemma~\ref{lem:pk-unpredictability} the distribution of
  public keys generated from $S$ is unpredictable. The
  function $\GenAddr$ samples a public key from $S$ and applies the
  random oracle $H$ to it. Applying
  Lemma~\ref{lem:ro-unpredictability}, we obtain that
  $X \cind \uniform(\{0, 1\}^\kappa)$.

  The function $H'(x) = H(x) \xor 1$ is a random oracle (despite not
  being independent from the random oracle $H$).
  Since $\mathcal{T}$ is unpredictable, and
  applying Lemma~\ref{lem:ro-unpredictability} with random oracle $H'$, we
  obtain that $Y \cind \uniform(\{0, 1\}^\kappa)$.

  By transitivity, $X$ and $Y$ are computationally indistinguishable.
\end{proof}

From the above, we conclude that the tags used during the burn process must be
unpredictable. If the tag is chosen to contain a randomly generated public key
from a secure signature scheme, or its hash,
Lemmas~\ref{lem:pk-unpredictability}~and~\ref{lem:ro-unpredictability} show that
sufficient entropy exists to ensure uncensorability. Our cross-chain application
makes use of this fact.
