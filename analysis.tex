\section{Analysis}

We now move on to the analysis of our scheme. As the scheme is deterministic,
its correctness is straightforward to show.

\begin{theorem}[Correctness]
  The Proof-of-Burn protocol $\Pi$ of Section~\ref{section:construction} is \emph{correct}.
\end{theorem}
\begin{proof}
  We need to prove that $\forall \kappa, t. \BurnVerify(1^\kappa, t, \GenBurnAddress(1^\kappa, t)) = \textsf{true}$.

  Based on Algorithm~\ref{alg.construction}, $\BurnVerify(1^\kappa, t, \GenBurnAddress(1^\kappa, t)) = \textsf{true}$ if and only if $\GenBurnAddress(1^\kappa, t) = \GenBurnAddress(1^\kappa, t)$, which always holds as $\GenBurnAddress$ is deterministic.
\end{proof}

We now state a simple lemma pertaining to the distribution of Random Oracle
outputs.

\begin{lemma}[Perturbation]
  \label{lem.perturbation}
  Let $p(\kappa)$ be a polynomial and
  $F: \{0,1\}^\kappa \longrightarrow \{0,1\}^\kappa$ be a permutation.
  Consider the process which samples $p(\kappa)$ strings $s_1, s_2, \dots, s_{p(\kappa)}$ uniformly at random from the set $\{0, 1\}^\kappa$. The probability that there exists $i \neq j$ such that $s_i = F(s_j)$ is negligible in $\kappa$.
\end{lemma}
\begin{proof}
  Let \textsc{Match} denote the event that there exist $1 \leq i \neq j \leq p(\kappa)$ such that $s_i = F(s_j)$.
  Let $\textsc{Match}_{i, j}$ denote the event that $s_i = F(s_j)$. Apply a union bound to obtain

  \begin{align*}
    \Pr[\bigcup_{i, j}\textsc{Match}_{i, j}] &\leq \sum_{i, j} \Pr[\textsc{Match}_{i, j}] \\
    \Pr[\textsc{Match}] &\leq \sum_{i, j} \Pr[\textsc{Match}_{i, j}]
  \end{align*}

  But $\Pr[\textsc{Match}_{i, j}] = 2^{-p(\kappa)}$ and therefore
  $\Pr[\textsc{Match}] \leq \sum_{i \neq j} 2^{-p(\kappa)} \leq p^2(\kappa) 2^{-p(\kappa)}$.
\end{proof}

We will now apply the above lemma to show that our scheme is unspenable.

\begin{theorem}[Unspendability]
  If $H$ is a \emph{Random Oracle}, then the protocol $\Pi$ of Section~\ref{section:construction} is \emph{unspendable}.
\end{theorem}
\begin{proof}
  Let $\mathcal{A}$ be an arbitrary probabilistic polynomial time \textsc{spend-attack} adversary.
  The adversary $\mathcal{A}$ makes at most a polynomial number of queries $p(\kappa)$ to the Random Oracle.
  Consider the event \textsc{Match} of Lemma~\ref{lem.perturbation}.

  If the adversary is successful then it has presented $t, pk, pkh$ such that $H(pk) = pkh$ and $H(t) \xor 1 = pkh$.

  We observe that $\textsc{spend-attack}_{\mathcal{A}, \Pi}(\kappa) = \true \Rightarrow \textsc{Match}$.

  Therefore $\Pr[\textsc{spend-attack}_{\mathcal{A}, \Pi}(\kappa)] \leq Pr[\textsc{Match}]$. Applying Lemma~\ref{lem.perturbation} for the permutation $F(x) = x \xor 1$,
  we obtain
  $\Pr[\textsc{spend-attack}_{\mathcal{A}, \Pi}(\kappa)] \leq negl(\kappa)$.
\end{proof}

We note that the security of the signature scheme is not needed to prove unspendability. Were the signature scheme of the underlying cryptocurrency ever found to be \emph{forgeable}, the coins burned through our scheme would remain unspendable. We additionally remark that the
choice of the permutation $F(x) = x \xor 1$ is arbitrary. Any one-to-one
function beyond the identity function would work equally well.

Next, our binding theorem only requires that the hash function used is collision
resistant and is in the standard model.

\import{./}{algorithms/alg.collision-resistance-adversary.tex}

\begin{theorem}[Binding]
  If $H$ is a \emph{collision resistant} hash function then the protocol of Section~\ref{section:construction} is \emph{binding}.
\end{theorem}
\begin{proof}
  Let $\mathcal{A}$ be an arbitrary adversary against $\Pi$.
  We will construct the Collision Resistance adversary $\mathcal{A}^*$ against $H$.

  The collision resistance adversary, illustrated in Algorithm~\ref{alg.collision-resistance-adversary}, calls $\mathcal{A}$ and obtains two outputs, $t$ and $t'$. If $\mathcal{A}$ is successful then $t \neq t'$ and $H(t) \xor 1 = H(t') \xor 1$. Therefore $H(t) = H(t')$.

  We thus conclude that $\mathcal{A^*}$ is successful in the \textsf{Collision} game if and only if $\mathcal{A}$ is successful in the \textsf{Bind} game.

  \[
    \Pr[\textsf{Bind}_{\mathcal{A},\Pi} = \true]
    =
    \Pr[\textsf{Collision}_{\mathcal{A}^*,H} = \true]
  \]

  From the collision resistance of $H$ it follows that $\Pr[\textsf{Collision}_{\mathcal{A}^*,H} = 1] < negl(\kappa)$. Therefore,
  $\Pr[\textsf{Bind}_{\mathcal{A},\Pi} = 1] < negl(\kappa)$, so
  the protocol $\Pi$ is binding.
\end{proof}

% TODO: This lemma seems useless... We need to argue about predictability of Gen() by the adversary.
\begin{lemma}[Distinct keys]\label{lem:distinct-keys}
  Let $S = (\Gen, \Sig, \Ver)$ be a secure signature scheme and $p$ be any polynomial. Consider the process which calls $(pk_i, sk_i) \gets \Gen(1^\kappa)$ repeatedly and independently $p(\kappa)$ times to obtain $p(\kappa)$ samples. Then for all $i, j \in [p]$ with $i \neq j$, we have that $pk_i \neq pk_j$, except with negligible probability in $\kappa$.
\end{lemma}
\begin{proof}
  Consider the above process and let \textsc{repeat-key} be the event of two samples $i \neq j$ repeating, i.e., $pk_i = pk_j$. Then consider the following existential forgery adversary $\mathcal{A}$ for the signature scheme $S$. The adversary receives a public key $pk$ and attempts to forge a signature $\sigma$. The adversary calls $(sk', pk') \gets \Gen(1^\kappa)$ to generate a new key $(sk', pk')$. If $pk' \neq pk$, the adversary aborts. Otherwise, the adversary uses $sk'$ to create a signature forgery.

  If $pk = pk'$ then by the correctness of the signature scheme the forgery will be successful. Note that \textsc{repeat-key} is the event of any two keys being repeated among $p$ keys. Additionally, together the forgery challenger $\textsf{Sig-forge}$ and the adversary $\mathcal{A}$ independently generate a pair of keys. Then by applying a union bound we obtain that $p^2 \Pr[\textsf{Sig-forge}^{cma}_{\mathcal{A},S}] \geq \Pr[\textsc{repeat-key}]$. From the fact that $p$ is a polynomial and
  $\Pr[\textsf{Sig-forge}^{cma}_{\mathcal{A},S}]$ is negligible, it follows that $\Pr[\textsc{repeat-key}]$ is negligible.
\end{proof}

Recall that a distribution ensemble is \emph{unpredictable} if no
polynomial-time adversary can guess its output. The cryptographic
predictability game is illustrated in Algorithm~\ref{alg.predict-game}.

\import{.}{./algorithms/alg.predict-game.tex}

\begin{definition}[Unpredictable distribution]
  A distribution ensemble $\{X_{\kappa}\}_{\kappa\in\mathbb{N}}$ is
  \emph{unpredictable} if for all probabilistic polynomial-time adversaries
  $\mathcal{A}$ there exists a negligible function $negl$ such that
  $\Pr[\textsc{predict}_{\mathcal{A},X}(\kappa) = \true] < negl(\kappa)$.
\end{definition}

\begin{lemma}[Negligible predictability]
  Consider a distribution ensemble $\{X_{\kappa}\}_{\kappa\in\mathbb{N}}$ and
  a negligible function $negl(\kappa)$. If for all $\kappa$ for all
  $x \in [X_\kappa]$ we have that
  $\Pr_{x^* \gets X_\kappa}[x^* = x] \leq negl(\kappa)$, then
  $X$ is unpredictable.
\end{lemma}
\begin{proof}
  Consider a probabilistic polynomial-time adversary $\mathcal{A}$ which
  predicts $X_\kappa$. The adversary is not given any input beyond $1^\kappa$,
  hence the distribution of its output is independent from the choice of the
  challenger. Therefore

  \begin{align*}
  \Pr[\textsc{predict}_{\mathcal{A},X}(\kappa) = \true]
  &= \sum_{x' \in [X]}\Pr_{x \gets X}[\mathcal{A}(\kappa) = x'\land x = x']\\
  &= \sum_{x' \in [X]}\Pr[\mathcal{A}(\kappa) = x']\Pr_{x \gets X}[x = x']\\
  &\leq negl(\kappa)\sum_{x' \in [X]}\Pr[\mathcal{A}(\kappa) = x']
  \leq negl(\kappa)\\
  \end{align*}
\end{proof}

We now posit that no adversary can predict the public key of a secure signature scheme, except with negligible probability.

\begin{lemma}[Public key unpredictability]\label{lem:pk-unpredictability}
  Let $S = (\textsf{Gen}, \textsf{Sig}, \textsf{Ver})$ be a secure signature scheme.
  Then the distribution $X = \{(sk, pk) \gets \textsf{Gen}; pk\}$ is
  unpredictable.
\end{lemma}
\begin{proof}
  Consider an adversary $\mathcal{A}$ against the \textsc{predict} game and let $q$ denote its probability of success. Then modify the challenger game to invoke $\textsf{Gen}(1^\kappa)$ \emph{twice} independently after the adversary is invoked to obtain $pk'_1$ and $pk'_2$. Consider the event $\textsc{predict-both}$ in which the adversary's guess $pk$ matches the result of both invocations of $\textsf{Gen}$, i.e., $pk'_1 = pk$ and $pk'_2 = pk$.
  Because the invocations of $\textsf{Gen}$ are independent, the probability of the adversary succeeding in guessing \emph{both} invocations of $\textsf{Gen}$ is $q^2$. Therefore
  $
  \Pr[\textsf{predict}_{\mathcal{A},X}(\kappa)]^2 = \Pr[\textsc{predict-both}]
  $.

  Conditioned on the event $\textsc{predict-both}$, we have $pk'_1 = pk'_2$. This is an execution which invokes $\textsf{Gen}$ exactly twice and tries to find non-unique keys in these invocations. Applying Lemma~\ref{lem:distinct-keys} with $p = 2$, we obtain that
  $
  \Pr[\textsc{predict-both}] \leq \textsf{negl}(\kappa)
  $.
  Therefore
  $
  \Pr[\textsf{predict}_{\mathcal{A},X}(\kappa)] \leq \textsf{negl}(\kappa)
  $.
\end{proof}

\import{.}{./algorithms/alg.predictability-adversary.tex}

The following theorem shows that the output of the random oracle is
indistinguishable from random if the input is unpredictable. For reference, the
definition of computational indistinguishability is included in
Appendix~\ref{sec.comp-ind}.

\begin{lemma}[Random Oracle unpredictability]\label{lem:ro-unpredictability}
  Let $\mathcal{T}$ be an unpredictable distribution ensemble and $H$ be a
  Random Oracle.
  The distribution $X = \{t \gets \mathcal{T}; H(t)\}$ is indistinguishable from
  the uniform distribution $\uniform(\{0, 1\}^\kappa)$.
\end{lemma}
\begin{proof}
  Let $\mathcal{A}$ be an arbitrary polynomial distinguisher between
  $X$ and $\uniform(\{0, 1\}^\kappa)$.
  We construct an adversary $\mathcal{A}^*$
  against $\textsf{predict}_{\mathcal{T}}$.
  Let $r$ denote the (polynomial)
  maximum number of random oracle queries of $\mathcal{A}$.
  The adversary $\mathcal{A}^*$ is illustrated in
  Algorithm~\ref{alg.predictability-adversary} and works as follows.
  Initially, it chooses a random bit $b \stackrel{\$}{\gets} \{0, 1\}$ and
  sets $Z = X$ if $b = 0$, otherwise
  sets $Z = \uniform(\{0, 1\}^\kappa)$.
  It samples $z \gets Z$.
  If $b = 0$, then $z$ is chosen by applying \textsf{GenAddress} which involves
  calling the random oracle $H$ with some input $pk$.
  It then chooses one of $\mathcal{A}$'s queries $j \stackrel{\$}{\gets} [q]$
  uniformly at random. Finally, it outputs the input received by the random
  oracle during the $j^\text{th}$ query of $\mathcal{A}$.
  Consider the event $\query$ that $b = 0$ and $\mathcal{A}$ asks
  a random oracle query with input $pk$. Recall the computational
  indistinguishability game depicted in
  Algorithm~\ref{alg.dist-game} in which the distinguisher gives a guess $b^*$
  attempting to identify the origin $b$ of its input.

  If $b = 0$, then the distinguisher $\mathcal{A}$ receives a truly random input
  $pkh = H(pk)$.
  Consider the case where the distinguisher does not query the random oracle
  with input $pk$. In that case, the input of the distinguisher is truly random
  and therefore $\Pr[b^* = 0|b = 0|\lnot \query] = \Pr[b^* = 0|b = 1]$.

  Consider the case where $b = 0$ and apply total probability to obtain
  \begin{align*}
    &\Pr[b^* = 0|b = 0] =\\
    &\Pr[b^* = 0|\query]\Pr[\query] +
      \Pr[b^* = 0|b = 0|\lnot \query]\Pr[\lnot \query]\\
    \leq &\Pr[b^* = 0|\query]\Pr[\query] +
      \Pr[b^* = 0|b = 0|\lnot \query]\\
    \leq &\Pr[\query] + \Pr[b^* = 0|b = 0|\lnot \query]
  \end{align*}

  Then
  $
    \Pr[\textsc{dist-game}_{\mathcal{A},X,\uniform(\{0,1\}^\kappa)} = \true]
    =
    \Pr[b = b^*]
  $ is the probability of success of the distinguisher.
  Applying total probability we obtain

  \begin{align*}
    \Pr[b = b^*] &= \Pr[b = b^*|b = 0]\Pr[b = 0] + \Pr[b = b^*|b = 1]\Pr[b = 1]\\
                 &= \frac{1}{2}(\Pr[b^* = 0|b = 0] + \Pr[b^* = 1|b = 1])\\
                 &\leq \frac{1}{2}(\Pr[\query] + \Pr[b^* = 0|b = 0|\lnot \query]
                 + \Pr[b^* = 1|b = 1])\\
                 &= \frac{1}{2}(\Pr[\query] + \Pr[b^* = 0|b = 1]
                 + \Pr[b^* = 1|b = 1])\\
                 &= \frac{1}{2}(\Pr[\query] + \Pr[b^* = 0|b = 1]
                 + (1 - \Pr[b^* = 0|b = 1]))\\
                 &= \frac{1}{2}(1 + \Pr[\query]) \leq \frac{1}{2} + negl(\kappa)
  \end{align*}
\end{proof}

\begin{theorem}[Uncensorability]
  Let $S = (\Gen, \Sig, \Ver)$ be a \emph{secure signature scheme},
  $H$ be a \emph{Random Oracle},
  and $\mathcal{T}$ be an unpredictable tag distribution.
  Then the protocol of Section~\ref{section:construction} instantiated with
  $H, S, \mathcal{T}$ is \emph{uncensorable}.
\end{theorem}
\begin{proof}
  From Lemma~\ref{lem:pk-unpredictability} the distribution of
  public keys generated from $S$ form an unpredictable distribution. The
  function $\textsf{GenAddress}$ samples a public key from $S$ and applies the
  random oracle $H$ to it. Applying
  Lemma~\ref{lem:ro-unpredictability}, we obtain that
  $X \cind \uniform(\{0, 1\}^\kappa)$.

  The function $H'(x) = H(x) \xor 1$ is a random oracle (despite not
  being independent from the random oracle $H$).
  Since $\mathcal{T}$ is unpredictable, and
  applying Lemma~\ref{lem:ro-unpredictability} with random oracle $H'$, we
  obtain that $Y \cind \uniform(\{0, 1\}^\kappa)$.

  By transitivity, $X$ and $Y$ are computationally indistinguishable.
\end{proof}
