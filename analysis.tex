\section{Analysis}

\begin{theorem}[Correctness]
  The above Proof-of-Burn protocol $\Pi$ is \emph{correct}.
\end{theorem}
\begin{proof}
  TODO
\end{proof}

\begin{lemma}[Unspendability]
  Consider the construction of Algorithm~\ref{alg.construction} where
  $\textsf{RIPEMD160}(\cdot)$ is modelled as a Random Oracle, $\textsf{SHA256}$
  and $\textsf{SHA512}$  are collision-resistant. Then any address $\alpha$ for
  which a $pk$ is known such that $\alpha = \textsf{perturbe}(pk)$ is
  unspendable, the respective \emph{public key} under the Bitcoin address
  generation protocol cannot be found.
\end{lemma}
\begin{proof}(Sketch)
  From the fact that $pk$ is known.
  Suppose for contradiction that the public key was known.
  By the collision-resistance of $\textsf{SHA256}$, this gives a unique image of
  $k_2 = \textsf{SHA256}(\texttt{0x04} || pkr)$.
  The probability that, in a given execution, the Random Oracle will output two
  outputs $k_1, k_2$ with $k_1 \xor k_2 = \texttt{0x01}$ is negligible.
\end{proof}

% TODO: avoid double spending

\begin{theorem}
  If $H$ is modelled as a \emph{Random Oracle}, then the protocol $\Pi$ of Section~\ref{section:construction} is \emph{unspendable}.
\end{theorem}
\begin{proof}
  $H(t) \xor 1 = pkh$.
  Adversary knows a $pk$ st $H(pk) = pkh$.
  This is impossible under the Random Oracle model.
  TODO
\end{proof}

We note that the security of the signature scheme is not needed to prove unspendability. Were the signature scheme of Bitcoin ever found to be forgeable, the coins burned through our scheme would remain unspendable.

\begin{theorem}
  If $H$ is \emph{collision resistant} then the protocol of Section~\ref{section:construction} is \emph{binding}.
\end{theorem}
\begin{proof}
  Let $\mathcal{A}$ be an arbitrary adversary against $\Pi$.
  We will construct the Collision Resistance adversary $A^*$ against $H$.
  TODO

  \[
    \Pr[\textsf{Bind}_{\mathcal{A},\Pi} = 1]
    <
    \Pr[\textsf{Collision}_{\mathcal{A}^*,H} = 1]
  \]

  From the collision resistance of $H$ it follows that $\Pr[\textsf{Collision}_{\mathcal{A}^*,H} = 1] < negl(\kappa)$. Therefore,
  $\Pr[\textsf{Bind}_{\mathcal{A},\Pi} = 1] < negl(\kappa)$, so
  the protocol $\Pi$ is binding.
\end{proof}

\begin{theorem}
  If $H$ is modelled as a \emph{Random Oracle} and $S = (\Gen, \Sig, \Ver)$ is a \emph{secure signature scheme} then the protocol of Section~\ref{section:construction} is \emph{uncensorable}.
\end{theorem}
\begin{proof}
  Let $\mathcal{A}$ be an arbitrary adversary against $\Pi$.
  We will construct the Collision Resistance adversary $A^*$ against $H$.

  TODO

  From security of signature scheme, calling $Gen()$ twice will never yield the same $pk$.

  TODO

  \[
    \Pr[\textsf{Bind}_{\mathcal{A},\Pi} = 1]
    <
    \Pr[\textsf{Collision}_{\mathcal{A}^*,H} = 1]
  \]

  From the collision resistance of $H$ it follows that $\Pr[\textsf{Collision}_{\mathcal{A}^*,H} = 1] < negl(\kappa)$. Therefore,
  $\Pr[\textsf{Bind}_{\mathcal{A},\Pi} = 1] < negl(\kappa)$, so
  the protocol $\Pi$ is binding.
\end{proof}
