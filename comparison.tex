\section{Comparison}

We present a few alternatives for performing proof of burn which have been proposed in previous work. All of the schemes presented are captured by our burn protocol definition.

\noindent
\textbf{OP\_RETURN.}
Bitcoin natively employs a mechanism for burning money using a specialized
opcode called OP\_RETURN~\cite{bartoletti2017analysis}. Unfortunately,
creating an OP\_RETURN proof-of-burn transaction requires specialized software and thus is not
user-friendly. However, it does benefit the Bitcoin network by allowing the UTXO
to be pruned, at the cost of not being uncensorable.

\noindent
\textbf{P2SH OP\_RETURN.}
An OP\_RETURN script can also be used as the redeemScript for a P2SH~\cite{p2sh} address. The script is hashed and turned into an address. The scheme is unspendable since there is no scriptSig that could make an OP\_RETURN script succeed. Additionally it is uncensorable if the tag is not revealed. Finally, the scheme is user-friendly since the user can create a burn transaction using any regular wallet.

\noindent
\textbf{Nothing-up-my-sleeve address.}
An address is manually crafted, such that it is obvious that it has not been obtained by regular keypair generation through the secure signature scheme's \textsf{Gen}. For example, a public key hash of all zeros could be considered a nothing-up-my-sleeve address. For Bitcoin, nothing-up-my-sleeve addresses are usually created based on the base58 encoding. We remind the reader that base58 addresses are generated from the public key hash, where a magic byte, 0x00 is prepended to the public key hash before it is turned to address. Additionally, it is checksummed and the first 4 bytes of the checksum are appended to the public key hash before it is base58 encoded. The creation of a nothing-up-my-sleeve address happens as follows: the creator writes a string of their choosing in base58, starting with $1$ so that the magic 0x00 byte is guaranteed to be at the front. The decoded version must be exactly 21 bytes. Afterwards the checksum is calculated and its first 4 bytes appended, yielding the full 25-byte address which is then base58 encoded. Nothing-up-my-sleeve addresses have been used in Counterparty~\cite{counterparty}. The scheme is not binding, since it does not encode any tag as the address is constant.

We compare the aforementioned schemes on whether they satisfy the burn protocol properties we define: binding, unspendability, uncensorability. Additionally, we compare them based on how easily they translate to multiple cryptocurrencies. For instance, OP\_RETURN and P2SH OP\_RETURN rely on Bitcoin Script semantics and do not directly apply to any non-Bitcoin based cryptocurrencies like Ethereum, thus we say they are not flexible. The comparison is illustrated on the table that follows.

\begin{center}
    \newcommand{\y}{$\bullet$}
    \newcommand{\n}{}
    \begin{tabular}{ |c|c|c|c|c| }
     \hline
                                        & Binding & Flexible & Unspendable & Uncensorable \\
     \hline
     OP\_RETURN                         & \y      & \n       & \y          & \n \\
     P2SH OP\_RETURN                    & \y      & \n       & \y          & \y \\
     Nothing-up-my-sleeve address       & \n      & \y       & \y          & \y \\
     $a \xor 1$ \textbf{(this work)}    & \y      & \y       & \y          & \y \\
     \hline
    \end{tabular}
\end{center}
