\section{Introduction}\label{section:introduction}
Proof-of-burn was invented in 2012 by Iain Stewart~\cite{stewart} as a mechanism for irrevocably destroying cryptocurrency while being able to verifiably prove to other parties that it has been indeed destroyed. It has since been used as a consensus mechanism similar to proof-of-stake (Slimcoin~\cite{slimcoin}), as a mechanism for establishing identity (OpenBazaar~\cite{zindros2016trust}), notarization (Carbon dating~\cite{clark2012commitcoin} and OpenTimestamps~\cite{todd2016opentimestamps}) and bootstrapping a new cryptocurrency (Counterparty~\cite{counterparty}).

While its adoption is undeniable, there has not been a formal treatment for proof-of-burn. This is the gap this work aims to fill.

\noindent
\textbf{Workflow.}
A user who wishes to burn her coins generates an address which we call a \emph{burn address} that encodes some metadata (called the \emph{tag}). She can then proceed to send any amount of cryptocurrency to the burn address. After burning her cryptocurrency, she may prove to any interested party that she has actually irrevocably destroyed the cryptocurrency in question.

\begin{itemize}
    \item \textbf{Unspendability.} No one should be able to spend the amount burned on the source blockchain.
    \item \textbf{Binding.} The burn must commit to details of the receiver on the target blockchain.
    \item \textbf{Uncensorability.} Miners who do not agree with this scheme on the source blockchain should not be able to censor burn transactions.
    \item \textbf{Usability.} The user should be able to create a burn transaction using her regular source cryptocurrency wallet.
    \item \textbf{Miner-isolation.} No miner on the target blockchain should be required to connect to the network of any of the source blockchains. % TODO mv
\end{itemize}

We propose a system which allows decentralized proof-of-burn-based bootstrapping and satisfies all the desired properties. To our knowledge, we are the first to do so.

In the heart of our construction, we make use of a primitive called
Non-Interactive Proofs of Proof-of-Work (NIPoPoWs)~\cite{nipopows}, which allows the
construction of cross-chain certificates that can later be embedded in a remote
blockchain. The remote blockchains require to be \emph{interlinked}, a process
which can be performed using a velvet fork~\cite{velvet}. These interlinks have
already been deployed on some testnets by velvet fork~\cite{gtklocker} as well
as on some coins from genesis~\cite{ergo}.

\noindent
\textbf{Previous work.}
Proof-of-burn has been proposed as a mechanism of consensus akin to
proof-of-stake (cf. Slimcoin and Factom), as well as a mechanism for
establishing identity (OpenBazaar~\cite{zindros2016trust}) and notarization
(Carbon dating~\cite{clark2012commitcoin} and
OpenTimestamps~\cite{todd2016opentimestamps}). For bootstrapping
cryptocurrencies, proof-of-burn has been previously used in systems where direct
observation allowed for the system to give rise to the new tokens such as
Counterparty~\cite{counterparty}, a velvet fork of Bitcoin. The construction posed by Counterparty only satisfies two of the desired properties, namely unspendability and binding. It does not satisfy usability as the user needs to create a special type of transaction with an \textsf{OP\_RETURN} output. Because of this it also does not satisfy uncensorability, since this form of transaction can be easily detected by miners. Finally, it does not satisfy miner-isolation as users need to be connected to the Bitcoin network in order for the burn transactions to be verified.

NIPoPoWs were introduced in~\cite{nipopows}, improving upon previous
work~\cite{popow,highway}. Cross-chain applications using NIPoPoWs include
proof-of-work sidechains~\cite{pow-sidechains}. Comparable schemes have been
proposed for proof-of-stake systems~\cite{pos-sidechains}.

\noindent
\textbf{Our contributions.}
A summary of our contributions is as follows:
\begin{enumerate}[wide, labelwidth=!, labelindent=0pt, label=(\roman*)]
    \item \textbf{Primitive definition.} Our definitional contribution introduces proof-of-burn as a cryptographic primitive for the first time. We
    define it as a protocol which consists of two algorithms, a burn address \emph{generator} and a burn address \emph{verifier}. We put forth the foundational properties which make for secure burn protocols, namely \emph{unspendability}, \emph{binding}, and \emph{uncensorability}.
    \item \textbf{Novel construction.} We propose a novel and simple construction which is flexible and can be adapted for use in existing cryptocurrencies, as long as they use public key hashes for address generation. To our knowledge, all popular cryptocurrencies are
    compatible with our scheme. We prove our construction secure in the Random Oracle model.
    \item \textbf{Experimental results.} We provide a compehensively tested production grade implementation of our burn verifier in Ethereum
    written in Solidity, which we release as open source software. Our implementation can be used to consume proofs of burn of a source blockchain
    within a target blockchain. We provide experimental measurements for the cost of burn verification and find that, in current Ethereum prices,
    burn verification costs $\$0.31$ per transaction.
    This allows coins burned on one blockchain to be consumed on another for the purposes of, for example, ERC-20 tokens creation~\cite{erc20}.
\end{enumerate}

We propose a cryptocurrency proof-of-burn bootstrapping mechanism which is both
decentralized and miner-isolated. Our construction in principle allows burning
from any proof-of-work-based cryptocurrency, but in particular we highlight the
ability to burn from Bitcoin~\cite{bitcoin},
Bitcoin Cash, Litecoin~\cite{lee2011litecoin},
Ethereum~\cite{buterin2014next,wood2014ethereum}, Ethereum
Classic~\cite{classic2017ethereum}, Monero~\cite{van2013cryptonote},
ZCash~\cite{SP:BCGGMT14,hopwood2016zcash}.

\noindent
\textbf{Notation.} We use $\uniform(S)$ to denote the uniform distribution
obtained by sampling any item of the set $S$ with probability $\frac{1}{|S|}$.
We denote the support of a distribution $\mathcal{D}$ by $[\mathcal{D}]$.
We denote the empty string by $\epsilon$ and string concatenation by $\conc$.
