\begin{abstract}
  \emph{Proof-of-burn} has been used as a mechanism to destroy cryptocurrency in a
  verifiable manner. Despite its well known use, the mechanism has not been
  previously formally studied as a primitive. In this paper, we put forth the
  first cryptographic definition of what a proof-of-burn protocol is.
  The burn protocol consists of two functions. First, a function
  which generates a cryptocurrency address. When a user sends money to this address,
  the money is irrevocably destroyed. Second, a verification
  function which checks that an address is really unspendable.
  We propose the following properties for burn protocols. \emph{Unspendability},
  which mandates that an address which verifies correctly as a burn address cannot be used for
  spending; \emph{binding}, which allows associating metadata with a particular burn;
  and \emph{uncensorability}, which mandates that a burn address is indistinguishable
  from a regular cryptocurrency address. Our scheme captures all previously
  defined proof-of-burn protocols.
  Next, we design a novel construction for burning which is simple
  and flexible, making it compatible with all existing popular cryptocurrencies.
  We prove our scheme is secure in the Random Oracle model.
  Finally, we explore the applications of burn in the context of destroying
  value in a legacy cryptocurrency system to bootstrap a new one.
  Using this application, a user can burn her coins in the source blockchain
  and subsequently create a proof-of-burn, a
  special string proving that the burn took place, which she can submit to
  the destination blockchain to be rewarded with the corresponding amount.
  The user can use
  any standard source blockchain wallet to conduct the burn without
  requiring specialized software, making our scheme user friendly.
  We adapt previous work to create a mechanism through which the target
  blockchain miners can verify that a burn has occurred in the source blockchain
  without requiring them to monitor it. Finally, we implement our burn scheme
  on the Ethereum blockchain and experimentally measure that the gas costs needed for
  verification are as low as standard Bitcoin transaction fees, illustrating
  that our scheme is practical.
\end{abstract}
