\section{Deployment to Bitcoin}\label{sec.real}
\import{./}{algorithms/alg.bitcoin-real.tex}

The scheme described above works for a generic P2PKH cryptocurrency and can be adapted to any cryptocurrency. We illustrate its suitability by giving a precise construction for Bitcoin, taking into account the engineering details that are behind the generation of a Bitcoin P2PKH address. A comparable approach can be used to generate Ethereum addresses or others.

The way Bitcoin generates P2PKH addresses is illustrated in Algorithm~\ref{alg.bitcoin-real}. Here, $\Gen$ generates an elliptic curve public key (of fixed key size $\kappa = 256$). After the elliptic curve public key is generated, it is marked by a magic number and subsequently hashed by the so-called $\BTCHASH$ algorithm, which consists of evaluating $\RIPEMD$ on the $\SHA$ of the public key. The resulting hash is additionally prefixed by a magic number indicating that the execution is taking place on the main net (and not the test net), and the final address, together with a checksum, are encoded using $\baseencode$ to obtain the final address.

Our burn algorithm follows the same structure for address generation, ensuring that the magic numbers and checksums validate correctly. In this construction, the hash function which is modelled as a random oracle is the $\BTCHASH$ algorithm. The algorithm is illustrated in Algorithm~\ref{alg.construction-real} and works as follows.
Given a tag $t$, the user derives a $160$-byte
hash $th = \RIPEMD(\SHA(t))$ which looks like a public key hash.
The least significant bit of
$th$ is then flipped to achieve unspendability. This produces the $20$-byte
\emph{perturbated hash} $th'$. The perturbated hash
is then prefixed with \texttt{0x00} to designate that we're working on the Bitcoin mainnet as usual. The checksum is calculated and appended
to it, and the result is
$\baseencode$ encoded into a Bitcoin address which correctly validates.

\import{./}{algorithms/alg.construction-real.tex}
