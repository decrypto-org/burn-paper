\section{Deployment to Real Bitcoin}\label{sec.real}
\import{./}{algorithms/alg.bitcoin-real.tex}

The algorithm works as follows. Initially, the user generates a fresh key
$(pk, sk)$ to which they will receive their newly generated tokens. From the $pk$, the user derives a $64$-byte
\emph{key reference} $pkr = \textsf{SHA512}(pk)$. The key reference is then
treated as an uncompressed Bitcoin or Bitcoin Cash public key pair, i.e., its
first $32$-byte part is considered the $X$ coordinate and the second $32$-byte
part is considered the $Y$ coordinate of an elliptic curve. The key reference is
then prefixed with \texttt{0x04} and the result is hashed using
$\textsf{RIPEMD160}(\textsf{SHA256}(\cdot))$ as is usual for address generation.
This produces the \emph{key hash reference} $pkh$. The least significant bit of
$pkh$ is then flipped. This step is imperative to the security of the scheme to
ensure that the generated address is unspendable. This produces the $20$-byte
\emph{perturbated key hash reference} $pkh'$. The perturbated key hash reference
is then prefixed with \texttt{0x00} for the Bitcoin mainnet or with
\texttt{0x6f} for the Bitcoin testnet as usual. The checksum of the prefixed
perturbated key hash reference is then calculated as the last $4$ bytes of
$\textsf{SHA256}(\textsf{SHA256}(prefix || pkh'))$. These $4$ bytes are appended
to the prefixed perturbated key hash reference as a suffix, and the result is
\textsf{base58} encoded into a Bitcoin address.

\import{./}{algorithms/alg.construction-real.tex}
