\section*{Appendix}

Our Appendix is structured as follows.
In Appendix~\ref{sec.consumption} we explore the application of destroying value in a legacy cryptocurrency to bootstrap a new one. The user burns coins in the source blockchain and subsequently creates a proof-of-burn, a short string proving that the burn took place, which she then submits to the destination blockchain to be rewarded with a corresponding amount. The user can use a standard wallet to conduct the burn without requiring specialized software, making our scheme user friendly. We propose burn verification mechanisms with different security guarantees, noting that the target blockchain miners do not necessarily need to monitor the source blockchain. 
In Appendix~\ref{sec.empirical}
we implement the verification of Bitcoin burns
as an Ethereum smart contract and experimentally measure that the gas costs needed for
verification are as low as standard Bitcoin transaction fees, illustrating
that our scheme is practical. In Appendix~\ref{sec.real} we show how
our scheme can be adopted to the practical implementation details of Bitcoin.
In Appendix~\ref{sec:proofs} we provide the formal definitions and proofs of our
claims. Finally, in Appendix~\ref{sec:standard}
we discuss potential future directions in relaxing the Random Oracle assumption
and propose a scheme for which we show some desirable properties in
the Common Random String model~\cite{STOC:BluFelMic88}.

\import{./}{consumption.tex}
\import{./}{empirical-results.tex}
\import{./}{real.tex}
\import{./}{proofs.tex}
\import{./}{standard.tex}
\import{./}{acknowledgements.tex}
