\section{Full proofs}

\subsection{Computational indistinguishability}\label{sec:comp-ind}

\import{.}{./algorithms/alg.dist-game.tex}

We review the definition of computational indistinguishability between two
distributions $X$ and $Y$. Define the cryptographic game illustrated in
Algorithm~\ref{alg.dist-game}.
Computational indistinguishability mandates that no adversary can win the game,
except with negligible probability.

\begin{definition}[Computational indistinguishability]
  Two distribution ensembles $\{X_\kappa\}_{\kappa\in\mathbb{N}}$ and $\{Y_\kappa\}_{\kappa\in\mathbb{N}}$ are
  \emph{computationally indistinguishable}
  if for every probabilistic polynomial-time adversary $\mathcal{A}$,
  there exists a negligible function $\negl$ such that
  $\Pr[\distattack_{\mathcal{A},X,Y}(\kappa) = \true] < \negl$.
\end{definition}

It is clear that applying an efficiently computable function to
indistinguishable distributions preserves indistinguishability.

\import{./}{algorithms/alg.pres-distinguisher.tex}

\begin{lem}[Indistinguishability preservation]\label{lem:ind-pres}
  Given two
  computationally indistinguishable distribution ensembles
  $\{X_\kappa\}_{\kappa\in\mathbb{N}}$ and $\{Y_\kappa\}_{\kappa\in\mathbb{N}}$,
  let
  $\{f_\kappa\}_{\kappa\in\mathbb{N}}$,
  be an efficiently computable family of functions $X_\kappa \longrightarrow Y_\kappa$.
  Then the distribution ensembles
  $X' = \{f_\kappa(X_\kappa)\}_{\kappa\in\mathbb{N}}$
  and
  $Y' = \{f_\kappa(Y_\kappa)\}_{\kappa\in\mathbb{N}}$
  are computationally indistinguishable.
\end{lem}
\begin{proof}
  Let $\mathcal{A}$ be a probabilistic polynomial-time distinguisher
  between $X'$ and $Y'$. Consider the probabilistic polynomial-time distinguisher $\mathcal{A}^*$ between $X$ and $Y$ illustrated in Algorithm~\ref{alg.pres-distinguisher}. Then
  $\Pr[\distattack_{\mathcal{A}^*}(\kappa) = \true] =
   \Pr[\distattack_{\mathcal{A}}(\kappa) = \true]$.
  As $\Pr[\distattack_{\mathcal{A}^*}(\kappa) = \true] \leq \frac{1}{2} + \negl$, therefore $\Pr[\distattack_{\mathcal{A}}(\kappa) = \true] \leq \frac{1}{2} + \negl$.
\end{proof}

\subsection{Unpredictable distributions}

We call a distribution ensemble \emph{unpredictable} if no
polynomial-time adversary can guess its output. The cryptographic
predictability game is illustrated in Algorithm~\ref{alg.predict-game} and the
security definition is given below.

\import{.}{./algorithms/alg.predict-game.tex}

\begin{definition}[Unpredictable distribution]
  A distribution ensemble $\{X_{\kappa}\}_{\kappa\in\mathbb{N}}$ is
  \emph{unpredictable} if for all probabilistic polynomial-time adversaries
  $\mathcal{A}$ there is a negligible function $\negl$ such that
  \[\Pr[\predict_{\mathcal{A},X}(\kappa) = \true] < \negl\,.\]
\end{definition}

We observe that, if each element of a distribution appears with negligible
probability, then the distribution must be unpredictable.

\begin{lem}[Negligible unpredictability]\label{lem:negl-unpred}
  Consider a distribution ensemble $\{X_{\kappa}\}_{\kappa\in\mathbb{N}}$ and
  a negligible function $\negl$. If
  \[\max_{x \in [X_\kappa]}\Pr_{x^* \gets X_\kappa}[x^* = x] \leq \negl\,,\]
  then $X$ is unpredictable.
\end{lem}
\begin{proof}
  Consider a probabilistic polynomial-time adversary $\mathcal{A}$ which
  predicts $X_\kappa$. The adversary is not given any input beyond $1^\kappa$,
  hence the distribution of its output is independent from the choice of the
  challenger. Therefore

  \begin{align*}
  \Pr[\predict_{\mathcal{A},X}(\kappa) = \true] &=
  \sum_{x' \in [X]}\Pr_{x \gets X}[\mathcal{A}(\kappa) = x'\land x = x'] =\\
  \sum_{x' \in [X]}\Pr[\mathcal{A}(\kappa) = x']\Pr_{x \gets X}[x = x']
  &\leq \negl\sum_{x' \in [X]}\Pr[\mathcal{A}(\kappa) = x']
  \leq \negl
  \,.
  \end{align*}
\end{proof}

% The converse of Lemma~\ref{lem:negl-unpred}
% also holds for distributions that are efficiently samplable.
%
% \import{./}{algorithms/alg.predictability-adversary-guess.tex}
%
% \begin{lem}[Unpredictable negligibility]\label{lem:unpred-negl}
%   Consider an efficiently samplable unpredictable distribution ensemble $\{X_\kappa\}_{\kappa\in\mathbb{N}}$. Then there exists a negligible function $\negl$ such that $\max_{x \in X_\kappa}\Pr_{x^* \gets X_\kappa}[x^* = x] \leq \negl$.
% \end{lem}
% \begin{proof}
%   Consider the probabilistic polynomial-time predictability adversary $\mathcal{A}$ which samples from $X_\kappa$ and returns its sample illustrated in Algorithm~\ref{alg.predictability-adversary-guess}. By the independence of the
%   sampling from $X_\kappa$ of the challenger and the adversary, we have that
%   $\Pr[\predict_{\mathcal{A},X}(\kappa) = \true]
%    = \sum_{x \in [X_\kappa]} \Pr^2_{x^* \gets X_\kappa}[x^* = x]$,
%   which is negligible from the assumption that $X$ is unpredictable. Hence
%   $\max_{x \in X_\kappa}\Pr_{x^* \gets X_\kappa}[x^* = x]$ is also negligible.
% \end{proof}
%
% We remark that it is instrumental to the above proof that the distribution ensemble $\{X_\kappa\}_{\kappa\in\mathbb{N}}$ is efficiently samplable; otherwise the statement does not hold.
%
Finally, we observe that public keys generated from secure signature schemes must
be unpredictable.

\restateLemPkUnpredictability*
\begin{proof}
  Let
  $p = \max_{\widehat{pk} \in [X_\kappa]}\Pr_{pk \gets X_\kappa}[pk = \widehat{pk}]$.
  Consider the existential forgery adversary $\mathcal{A}$ which works as
  follows. It receives $pk$ as its input from the challenger, but ignores it
  and generates a new key pair $(pk', sk') \gets \textsf{Gen}(1^\kappa)$.
  Since the
  two invocations of $\textsf{Gen}$ are independent,
  \begin{align*}
    \Pr[pk = pk'] \geq \max_{\widehat{pk} \in [X_\kappa]}\Pr[pk = \widehat{pk} \land pk' = \widehat{pk}]\\
  = \max_{\widehat{pk} \in [X_\kappa]}\Pr[pk = \widehat{pk}]\Pr[pk' = \widehat{pk}]\\
  = \max_{\widehat{pk} \in [X_\kappa]}\left(\Pr[pk = \widehat{pk}]\right)^2
  = p^2
  \,.
  \end{align*}

  The adversary checks
  whether $pk = pk'$. If not, it aborts. Otherwise, it uses $sk'$ to sign the
  message $m = \epsilon$ and returns the forgery $\sigma = \textsf{Sig}(sk, m)$.
  From the correctness of the signature scheme, if $pk = pk'$, then
  $\textsf{Ver}(pk, \textsf{Sig}(sk, m)) = \true$ and the adversary is
  successful. Since the signature scheme is secure,
  $\Pr[\textsf{Sig-forge}^{cma}_{\mathcal{A},S}] = \negl$.
  But $\Pr[pk = pk'] \leq \Pr[\textsf{Sig-forge}^{cma}_{\mathcal{A},S}]$ and
  therefore $p \leq \sqrt{\Pr[pk = pk']} \leq \negl$. Applying
  Lemma~\ref{lem:negl-unpred}, we deduce that the distribution ensemble $X_\kappa$ is
  unpredictable.
\end{proof}

\subsection{Random Oracle properties}

In this section, we state some statistical properties of the Random
Oracle, which are useful for the proofs of our main results.

\restateLemPerturbation*
\begin{proof}
  Let \textsc{Match} denote the event that there exist $1 \leq i \neq j \leq p(\kappa)$ such that $s_i = F(s_j)$.
  Let $\textsc{Match}_{i, j}$ denote the event that $s_i = F(s_j)$. Apply a union bound to obtain
  $
    \Pr[\bigcup_{i \neq j}\textsc{Match}_{i, j}] \leq \sum_{i \neq j} \Pr[\textsc{Match}_{i, j}]
  $.
  But $\Pr[\textsc{Match}_{i, j}] = 2^{-p(\kappa)}$ and therefore
  $\Pr[\textsc{Match}] \leq \sum_{i \neq j} 2^{-p(\kappa)} \leq p^2(\kappa) 2^{-p(\kappa)}$.
\end{proof}

\import{.}{./algorithms/alg.predictability-adversary.tex}

\restateLemRoUnpredictability*
\begin{proof}
  Let $\mathcal{A}$ be an arbitrary polynomial distinguisher between
  $X$ and $\uniform(\{0, 1\}^\kappa)$.
  We construct an adversary $\mathcal{A}^*$
  against $\predict_{\mathcal{T}}$.
  Let $r$ denote the (polynomial)
  maximum number of random oracle queries of $\mathcal{A}$.
  The adversary $\mathcal{A}^*$ is illustrated in
  Algorithm~\ref{alg.predictability-adversary} and works as follows.
  Initially, it chooses a random bit $b \stackrel{\$}{\gets} \{0, 1\}$ and
  sets $Z = X$ if $b = 0$, otherwise
  sets $Z = \uniformk$.
  It samples $z \gets Z$.
  If $b = 0$, then $z$ is chosen by applying $\GenAddr$ which involves
  calling the random oracle $H$ with some input $pk$.
  It then chooses one of $\mathcal{A}$'s queries $j \stackrel{\$}{\gets} [r]$
  uniformly at random. Finally, it outputs the input received by the random
  oracle during the $j^\text{th}$ query of $\mathcal{A}$.

  We will consider two cases. Either $\mathcal{A}$ makes a random oracle query
  containing $pk$, or it does not. We will argue that, if $\mathcal{A}$ makes
  a random oracle query containing $pk$ with non-negligible probability, then
  $\mathcal{A}^*$ will be successful with non-negligible probability. However,
  we will argue that, if $\mathcal{A}$ does not make the particular random
  oracle query, it will be unable to distinguish $X$ from $\uniformk$.

  Let $\query$ denote the event that $b = 0$ and $\mathcal{A}$ asks a random
  oracle query with input $pk$.
  Let $x$ denote the random variable sampled by the challenger in the
  predictability game of $\mathcal{A}^*$.
  Let $\extqry$ denote the event that $b = 0$ and $\mathcal{A}$ asks a
  random oracle query with input equal to $x$. Observe that, since the input to
  $\mathcal{A}$ does not depend on $x$, we have that
  $\Pr[\extqry] = \Pr[\query]$. As $j$ is chosen independently of the execution
  of $\mathcal{A}$, conditioned on $\extqry$ the probability that
  $\mathcal{A}^*$ is able to correctly guess which query caused $\extqry$ will
  be $\frac{1}{r}$. Therefore we obtain that
  $\Pr[\predict_{\mathcal{A}^*,\mathcal{T}}(\kappa) = \true]
   = \frac{1}{r}\Pr[\extqry]
   = \frac{1}{r}\Pr[\query]$.
  As
  $\Pr[\predict_{\mathcal{A}^*,\mathcal{T}}(\kappa) = \true] \leq \negl$
  and $r$
  is polynomial in $\kappa$, we deduce that $\Pr[\query] \leq \negl$.

  Consider the computational
  indistinguishability game depicted in
  Algorithm~\ref{alg.dist-game} in which the distinguisher gives a guess $b^*$
  attempting to identify the origin $b$ of its input.
  If $b = 0$, then the distinguisher $\mathcal{A}$ receives a truly random input
  $pkh = H(pk)$.
  If the distinguisher does not query the random oracle
  with input $pk$, the input of the distinguisher is truly random
  and therefore $\Pr[b^* = 0|b = 0|\lnot \query] = \Pr[b^* = 0|b = 1]$.

  Consider the case where $b = 0$ and apply total probability to obtain
  \begin{align*}
    &\Pr[b^* = 0|b = 0] =\\
    &\Pr[b^* = 0|\query]\Pr[\query] +
      \Pr[b^* = 0|b = 0|\lnot \query]\Pr[\lnot \query]\\
    \leq &\Pr[b^* = 0|\query]\Pr[\query] +
      \Pr[b^* = 0|b = 0|\lnot \query]\\
    \leq &\Pr[\query] + \Pr[b^* = 0|b = 0|\lnot \query]
  \end{align*}

  Then
  $
    \Pr[\distattack_{\mathcal{A},X,\uniformk} = \true]
    =
    \Pr[b = b^*]
  $ is the probability of success of the distinguisher.
  Applying total probability we obtain

  \begin{align*}
    \Pr[b = b^*] &= \Pr[b = b^*|b = 0]\Pr[b = 0] + \Pr[b = b^*|b = 1]\Pr[b = 1]\\
                 &= \frac{1}{2}(\Pr[b^* = 0|b = 0] + \Pr[b^* = 1|b = 1])\\
                 &\leq \frac{1}{2}(\Pr[\query] + \Pr[b^* = 0|b = 0|\lnot \query]
                 + \Pr[b^* = 1|b = 1])\\
                 &= \frac{1}{2}(\Pr[\query] + \Pr[b^* = 0|b = 1]
                 + \Pr[b^* = 1|b = 1])\\
                 &= \frac{1}{2}(\Pr[\query] + \Pr[b^* = 0|b = 1]
                 + (1 - \Pr[b^* = 0|b = 1]))\\
                 &= \frac{1}{2}(1 + \Pr[\query]) \leq \frac{1}{2} + \negl
  \end{align*}
\end{proof}
